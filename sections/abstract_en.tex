%% LaTeX2e class for student theses
%% sections/abstract_en.tex
%% 
%% Karlsruhe Institute of Technology
%% Institute for Program Structures and Data Organization
%% Chair for Software Design and Quality (SDQ)
%%
%% Dr.-Ing. Erik Burger
%% burger@kit.edu
%%
%% Version 1.2, 2016-09-20
%\documentclass{standalone}
%\begin{document}
%for data transmission and reception
\Abstract
Silicon dominates the field of computation and data processing, but its inherent inefficient performance in the optical domain has demanded the use of other semiconductor technologies, for light generation in telecommunication wavelengths, such as III-V indium phosphide (InP) compound semiconductors; and materials with strong intrinsic second-order nonlinearities that allow for electro-optic modulation. Silicon also has the best performance per price unit scale and thus, novel techniques must be developed in order to interconnect different devices required for high-speed, optical telecommunications transceivers into the silicon photonics platform. 

%The increasingly groundbreaking field of organic semiconductors provides better performance at higher data rates with lower driving voltages and better performance when compared to the nonlinearities of PN junction modulation. Due to the inherent incompatibilities at the atomic level between these two leading technologies, n
\par\medskip
A step forward into multi-platform chip integration was developed in this master's thesis. By using photonic wire bonding (PWB) as the link between InP-based laser sources and silicon-organic hybrid (SOH) electro-optic modulators on a silicon-on-insulator (SOI) platform, a step-by-step integration process flow is successfully realized and demonstrated. Further key improvements in the process are investigated and implemented to achieve homogeneous transmission and $V_\pi$ in an optically-integrated, 8-channel transmitter. 

\par\medskip
The best SOH MZM transmission achieved was \SI{-6.1}{\decibel} in a single channel, and the best PWB transmission of \SI{-4}{\decibel} with air as an overcladding was achieved. The homogeneous $V_\pi \cdot L$ obtained was \SI{1.3}{\volt\milli\meter}  Coherent detection is finally demonstrated in one module, featuring 16QAM modulation at 56 Gbaud, with 20\% FEC and BER of \SI{8e-3}, thus achieving a potential data transmission of 716 Gbit/s.
\par\medskip
%The outlook of this master thesis provides a streamlined process that opens the possibility to integrate new laser sources such as discrete mode lasers (DML), comb sources or edge-emitting lasers with better performance for higher-order data quadrature amplitude modulation (QAM) towards fully integrated, next-generation terabit optical transmitters.

%including optical path losses, half-wave voltage and data transmission performance; 

%Finally, an optically integrated, packaged device aiming for a commercially viable active optical cable (AOC) featuring 100G transfer data rate at a reasonable bit error ratio (BER) is proposed and analyzed.
%\end{document}