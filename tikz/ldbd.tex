% Flowcharting techniques for easy maintenance
% Author: Brent Longborough
\documentclass[x11names]{standalone}
\usepackage{tikz}
\usepackage{amsmath}
\usetikzlibrary{shapes,arrows,chains}
\begin{document}
% =================================================
% Set up a few colours
\colorlet{lcfree}{Green2}
\colorlet{lcnorm}{Blue3}
\colorlet{lccong}{Red3}
%\definecolor{lcfree}{rgb}{0.262,0.709,0.613}
%\definecolor{lcnorm}{rgb}{0.933, 0.227, 0.286}
%\definecolor{lccong}{rgb}{0.933, 0.756, 0.227}
% -------------------------------------------------
% Set up a new layer for the debugging marks, and make sure it is on
% top
\pgfdeclarelayer{marx}
\pgfsetlayers{main,marx}
% A macro for marking coordinates (specific to the coordinate naming
% scheme used here). Swap the following 2 definitions to deactivate
% marks.

\providecommand{\cmark}[2][]{\relax} 
\providecommand{\cmark}[2][]{%
  \begin{pgfonlayer}{marx}
    \node [nmark] at (c#2#1) {#2};
  \end{pgfonlayer}{marx}
  } 
% -------------------------------------------------
% Start the picture
\begin{tikzpicture}[%
    >=triangle 60,              % Nice arrows; your taste may be different
    start chain=going below,    % General flow is top-to-bottom
    node distance=6mm and 60mm, % Global setup of box spacing
    every join/.style={norm},   % Default linetype for connecting boxes
    ]
% ------------------------------------------------- 
% A few box styles 
% <on chain> *and* <on grid> reduce the need for manual relative
% positioning of nodes
\tikzset{
  base/.style={draw, on chain, on grid, align=center, minimum height=4ex},
  proc/.style={base, rectangle, text width=8em},
  test/.style={base, diamond, aspect=2, text width=5em},
  term/.style={proc, rounded corners},
  % coord node style is used for placing corners of connecting lines
  coord/.style={coordinate, on chain, on grid, node distance=6mm and 25mm},
  % nmark node style is used for coordinate debugging marks
  nmark/.style={draw, cyan, circle, font={\sffamily\bfseries}},
  % -------------------------------------------------
  % Connector line styles for different parts of the diagram
  norm/.style={->, draw, lcnorm},
  free/.style={->, draw, lcfree},
  cong/.style={->, draw, lccong},
  it/.style={font={\small\itshape}}
}
% -------------------------------------------------
% Start by placing the nodes
\node [proc, densely dotted, it] (p0) {goto \textbf{Home} Position: (0,0,0)};
% Use join to connect a node to the previous one 
\node [term, join,fill=lcfree!15]      {Request Ink Setup};
\node [proc, join] (p1) {goto Device $n$:\\ $(x_n/2,y_n,z_n)$};
\node [proc, join]      {goto \\ ($\mathbin{x_n +x^*_n, y_n +y_q}$)};
\node [proc, join]      {goto ($z_n -z_0 +z_m$)};
\node [test, join] (t1) {Needle in surface?};
% No join for exits from test nodes - connections have more complex
% requirements
% We continue until all the blocks are positioned
\node [proc] (p2)         {$z_0 + z_m$};
\node [proc, join] 	   {$x_n/2 + \Delta x \rightarrow y_n + y_\text{off}$};
\node [test, join] (t2) {Second Pass?};
\node [test, join] (t3) {Last Sample?};
\node [test, join] (t4) {$n \mathbin{{+}{=}} 1$};
% We position the next block explicitly as the first block in the
% second column.  The chain 'comes along with us'. The distance
% between columns has already been defined, so we don't need to
% specify it.
% Some more nodes specifically positioned (we could have avoided this,
\node [proc, join=by cong, right=of t4] (p9) {Print Report};
\node [term, join] (p10) {Exit Program};
% -------------------------------------------------
% Now we place the coordinate nodes for the connectors with angles, or
% with annotations. We also mark them for debugging.
\node [coord, left=of t4]  (c4)  {}; \cmark{4}   
\node [coord, right=of t4] (c4r) {}; \cmark[r]{4}  
% -------------------------------------------------
% A couple of boxes have annotations
% -------------------------------------------------
% All the other connections come out of tests and need annotating
% First, the straight north-south connections. In each case, we first
% draw a path with a (consistently positioned) annotation node, then
% we draw the arrow itself.
\path (t1.south) to node [near start, xshift=1em] {$y$} (p2);
  \draw [*->,lcnorm] (t1.south) -- (p2);
% ------------------------------------------------- 
% Now the straight east-west connections. To provide consistent
% positioning of the test exit annotations, we have positioned
% coordinates for the vertical part of the connectors. The annotation
% text is positioned on a path to the coordinate, and then the whole
% connector is drawn to its destination box.
\path (t4.east) to node [yshift=-1em] {$n=N$} (c4r); 
  \draw [o->,lcnorm] (t4.east) -- (p9);
% -------------------------------------------------
% Finally, the twisty connectors. Again, we place the annotation
% first, then draw the connector
\path (t4.west) to node [yshift=-1em] {$n<N$} (c4); 
  \draw [*->,lcnorm] (t4.west) -- (c4) |- (p1);
% -------------------------------------------------
% A last flourish which breaks all the rules
\draw [->,MediumPurple4, dotted, thick, shorten >=1mm]
  (p9.south) -- ++(5mm,-3mm)  -- ++(27mm,0) 
  |- node [black, near end, yshift=0.75em, it]
    {(Clear all settings and restart program)} (p0);
% -------------------------------------------------
\end{tikzpicture}
% =================================================
\end{document}